\chapter{Writing interface definitions}

\section{Basic structure}

The specification in an IDL file describes one or more interfaces,
which in turn may contain one or more methods. Here is an example:

\begin{center}\begin{tabular}{l@{\hspace{.4cm}}|@{\hspace{.5cm}}l}
\begin{minipage}{7cm}\small\begin{verbatim}
module storage {
   interface textfile {
      void readln(
        inout short pos, 
        out string line
      );
      void writeln(
        inout short pos,
        in string line
      );
      int get_pos();  
   };
};
\end{verbatim}\end{minipage} & 
\begin{minipage}{7cm}\small\begin{verbatim}
library storage {
   interface textfile {
      void readln(
        [in, out] short *pos, 
        [out, string] char **line
      );
      void writeln(
        [in, out] short *pos,
        [in, string] char **line
      );
      int get_pos();  
   };
};
\end{verbatim}\end{minipage} \\
\end{tabular}\end{center}

\IDL supports two specification languages, CORBA IDL and DCE
IDL, so most example code is shown in both languages; the CORBA code
is always on the left side.

Note that the syntax is very similar to C/C++, especially in the case
of DCE IDL. The most important difference is that every parameter
has a directional attribute (\texttt{in}, \texttt{inout} or \texttt{out}).
This is used to indicate whether the parameter contains

\begin{itemize}
\item input data, which must be copied from client to server,
\item output data, which is returned by the server, or
\item a combination of both
\end{itemize}

Another difference is the presence of special data types like 
\texttt{string}, which have additional semantics; for example, \texttt{string}
stands for a sequence of characters with a trailing zero, and 
\texttt{sequence} denotes an array of variable length. These special
types are explained later in this chapter.

Finally, it is possible to avoid naming conflicts by putting interfaces
into modules. For example, a network driver and a postal application
might both want to provide a \texttt{packet} interface, possibly with
different methods; this conflict can be resolved by using two separate
modules.

\section{Types}

\subsection{Basic types}

\subsubsection{Integers}

\IDL supports \texttt{word} and \texttt{signed\_word} types, which
are unsigned and signed integer types of machine word size.
They correspond to the word types defined by L4, and should be
used whenever possible.

Moreover, the following CORBA integer types are also supported:

\begin{center}\begin{tabular}{|l|l|l|}
\hline
Type name	& Size		& Value range			\\\hline
unsigned short 	& 16 bit	& 0..65535			\\
short 		& 16 bit 	& -32768..32767			\\
unsigned long	& 32 bit	& 0..4294967295			\\
long		& 32 bit	& -2147483648..2147483647 	\\	
unsigned long long & 64 bit 	& 0..18446744073709551615	\\
long long	& 64 bit	& -9223372036854775808..9223372036854775807 \\
\hline
\end{tabular}\end{center}

The type \texttt{int} is also supported, although it is deprecated
because the standard does not define its size (32 or 64 bit). Currently,
it has 32 bit on all platforms.

\subsubsection{Floating point}

The following floating point types are available:

\begin{center}\begin{tabular}{|l|l|l|l|}
\hline
Type name	& Size		& Value range		& Precision 	\\
\hline
float		& 32 bit	& $3.403\cdot 10^{38}$ 	& 6 digits	\\
double		& 64 bit	& $1.798\cdot 10^{308}$	& 15 digits	\\
long double	& 80 bit	& $1.1897\cdot 10^{4932}$ & 18 digits	\\
\hline
\end{tabular}\end{center}

\subsubsection{Characters}

\IDL supports both the 8-bit \texttt{char} and the 16-bit \texttt{wchar}
data types. The type \texttt{unsigned char} may also be used, but it is
deprecated. The reason is that in CORBA IDL, characters have special
semantics; for example, they might be translated to a different character
set by the marshalling code.

If you need an 8-bit data type for binary data, you should use
the \texttt{octet} type.

\subsubsection{Thread IDs}

\IDL has a special type named \texttt{threadid}, which corresponds to the
L4 thread ID type. For threads implementing an \IDL interface, the
interface name or the CORBA \texttt{Object} type may be used instead.

\subsubsection{Flexpages}

\IDL supports the L4 mapping primitives by providing a special type
named \texttt{fpage}. This type corresponds to the flexpage type of
the respective kernel interface; its size is platform-dependent.

\subsubsection{Miscellaneous}

\begin{center}\begin{tabular}{|l|l|l|}
\hline
Type name	& Size		& Possible values		\\\hline
boolean		& 1 bit		& \texttt{true}, \texttt{false}	\\
octet		& 8 bit		& 0..255			\\
void		& undefined	& none				\\
\hline
\end{tabular}\end{center}

The \texttt{void} data type may only be used for return values or
as the base type for a pointer. The \texttt{any} type, which is
also defined by CORBA, is not supported.

\subsection{Constructed types}

\subsubsection{Alias types}

You can use \texttt{typedef} to create your own types:

\begin{center}\begin{tabular}{l@{\hspace{.4cm}}|@{\hspace{.5cm}}l}
\begin{minipage}{7cm}\small\begin{verbatim}
typedef unsigned short word_t;
typedef string<40> max40char_t;
typedef long array_t[4][3];
\end{verbatim}\end{minipage} & 
\begin{minipage}{7cm}\small\begin{verbatim}
typedef unsigned short word_t;
/* Bounded strings not supported */
typedef long array_t[4][3];
\end{verbatim}\end{minipage} \\
\end{tabular}\end{center}

\IDL maps the types in an interface description to the target language
and adds a definition to the header files it produces. Thus, you can
also use these types in your own code.

\subsubsection{Sequences}

CORBA provides a special type for transferring variable-length data,
the \texttt{sequence} type. A sequence has a base type (e.g. \texttt{char})
and, optionally, a size bound. Consider the following example:

\begin{center}\begin{tabular}{l@{\hspace{.4cm}}|@{\hspace{.5cm}}l}
\begin{minipage}{7cm}\small\begin{verbatim}
typedef sequence<float, 7> some_t;
typedef sequence<char> another_t;
\end{verbatim}\end{minipage} & 
\begin{minipage}{7cm}\small\begin{verbatim}
/* DCE IDL does not support
   the sequence type */
\end{verbatim}\end{minipage} \\
\end{tabular}\end{center}

The first line defines an array of \texttt{float}s that does not contain
more than seven members (but may contain less); the second line defines
a character array of arbitrary size. Note that sequences may only 
appear in \texttt{typedef}s, not directly as an argument type.

When using sequences, please consider that \IDL needs to preallocate 
buffer space for them. Providing a tight size bound saves memory and
considerably improves performance.

Note that the sequence mapping in \IDL differs from the one specified 
in \cite{corba-clm}. In the \IDL mapping, the programmer is \emph{always}
responsible for the storage allocated for output sequences; the release
flag is not supported. Also, output sequence parameters do not use double
indirection; instead, they are treated just like ordinary structs.

\subsubsection{Arrays}

In DCE, there is no single type for variable-length arguments. Instead,
size and location of the data are specified independently.
Consider the following example:

\begin{center}\begin{tabular}{l@{\hspace{.4cm}}|@{\hspace{.5cm}}l}
\begin{minipage}{7cm}\small\begin{verbatim}
/* CORBA does not support
   the length_is attribute */
   
   
   
   
   
   
   
\end{verbatim}\end{minipage} & 
\begin{minipage}{7cm}\small\begin{verbatim}
interface foo {
  void bar(
    [in] int len, 
    [in, length_is(len)] float *addr
  );
  void xyz(
    [in, length_is(5)] short *addr
  );
};
\end{verbatim}\end{minipage} \\
\end{tabular}\end{center}

The stub code for \texttt{bar} transfers \texttt{len} floating point
numbers, starting at \texttt{addr}, whereas the code for \texttt{xyz}
always transfers five 16-bit integers.

On the server side, buffer space is allocated and freed by the stub 
code. In particular, for \texttt{out} arguments, the stub preallocates
enough buffer space and passes a pointer to it when the function
is invoked. Nevertheless, your code is not required to use this buffer;
it can save one copy operation by returning a pointer of its own.

On the client side, the stub allocates buffers for output values,
but does not free them. It is your responsibility to invoke
\texttt{CORBA\_free()} for every array that is returned by the server.

Note that the argument to \texttt{length\_is} denotes the number of 
elements, \emph{not} the size in bytes!

\subsubsection{Structs}

Structs are used exactly like their counterparts in C/C++:

\begin{center}\begin{tabular}{l@{\hspace{.4cm}}|@{\hspace{.5cm}}l}
\begin{minipage}{7cm}\small\begin{verbatim}
struct some_t {
  short a[4], b;
  float c;
};
\end{verbatim}\end{minipage} & 
\begin{minipage}{7cm}\small\begin{verbatim}
struct some_t {
  short a[4], b;
  float c;
};
\end{verbatim}\end{minipage} \\
\end{tabular}\end{center}

The stub allocates and frees buffers for the server; on the client
side, this is the responsibility of the user. However, 
\texttt{CORBA\_alloc()} is not used because structs have a fixed size;
instead, for an \texttt{out} parameter, the user supplies a pointer to 
an existing struct, which is then overwritten by the stub.

\subsubsection{Unions}

Unlike C/C++-style unions, a CORBA-compliant union needs a special
member, the discriminant, which is used to decide which type the union
currently contains. This is important because different types are
usually marshalled differently. Consider the following example:

\begin{center}\begin{tabular}{l@{\hspace{.4cm}}|@{\hspace{.5cm}}l}
\begin{minipage}{7cm}\small\begin{verbatim}
union U switch (int) {
  case 1 : long x;
  case 2 : 
  case 3 : string s;
  default: char c;
};
\end{verbatim}\end{minipage} & 
\begin{minipage}{7cm}\small\begin{verbatim}
union U switch (int a) {
  case 1 : long x;
  case 2 :
  case 3 : string s;
  default: char c;
};
\end{verbatim}\end{minipage} \\
\end{tabular}\end{center}

If the discriminant, which is always named \texttt{\_d} in CORBA, has 
the value 1, then one 32-bit value needs to be copied. When \texttt{\_d} 
is 2 or 3, however, the union contains a pointer, which must be 
dereferenced, resulting in a string to be transferred.

\emph{Note: Unions are not yet available in the current \IDL release!}
However, C/C++-style unions are supported. Such a union is always
copied directly and entirely; thus, it may not contain types with
special semantics, such as strings.

\subsubsection{Strings}

CORBA supports 8-bit and 16-bit strings with an optional length bound;
strings are always terminated by the value zero. Consider the
following example:

\begin{center}\begin{tabular}{l@{\hspace{.4cm}}|@{\hspace{.5cm}}l}
\begin{minipage}{7cm}\small\begin{verbatim}
typedef string<20> s20_t;

interface foo {
  void bar(
    in string a,
    inout s20_t b,
    out wstring<40> c
  );
};
\end{verbatim}\end{minipage} & 
\begin{minipage}{7cm}\small\begin{verbatim}
/* No bounded strings in DCE */

interface foo {
  void bar(
    [in, string] char *a,
    [in, out, string] char **b,
    [out, string] short **c
  );
};    
\end{verbatim}\end{minipage} \\
\end{tabular}\end{center}

The first argument to \texttt{foo::bar} is an 8-bit string of arbitrary
length, whereas the second argument may contain at most 20 characters;
the third argument is a 16-bit string of not more than 40 elements.
The first two arguments are terminated by a zero byte, the third one
ends with a 16-bit zero value.

On the server side, strings are managed by the stubs. For output
values, sufficient buffer space is allocated before the method is
invoked; however, the implementation is free to move the pointer to
another buffer. On the client side, the stubs allocate output buffers 
using \texttt{CORBA\_alloc()}, but do not free them; it is the 
responsibility of the user to invoke \texttt{CORBA\_free()} for each
one.

\subsubsection{Fixed point}

CORBA includes a special type for fixed-point, BCD-coded numbers.
This type is not supported by \IDL.

\subsubsection{Bitfields}

\IDL supports C/C++-style bitfields. These are allowed neither by
CORBA nor by DCE and should be used with care, because they are
highly platform-dependent. Consider the following example:

\begin{center}\begin{tabular}{l@{\hspace{.4cm}}|@{\hspace{.5cm}}l}
\begin{minipage}{7cm}\small\begin{verbatim}
struct msgdope_t {
  long cc     : 8;
  long parts  : 5;
  long mwords : 19;
};
\end{verbatim}\end{minipage} & 
\begin{minipage}{7cm}\small\begin{verbatim}
struct msgdope_t {
  long cc     : 8;
  long parts  : 5;
  long mwords : 19;
};
\end{verbatim}\end{minipage} \\
\end{tabular}\end{center}

\section{Exceptions}

Currently, only the CORBA exception handling is supported by \IDL.
In CORBA, there are two classes of exceptions: system and user-defined.
System exceptions may be raised by any method (e.g. when the IPC fails
or a timeout happens), whereas user-defined exceptions must be explicitly
specified by the interface description. Consider the following example:

\begin{center}\begin{tabular}{l@{\hspace{.4cm}}|@{\hspace{.5cm}}l}
\begin{minipage}{7cm}\small\begin{verbatim}
interface file {
  exception access_denied {};
  exception not_found {};

  void open(in string name)
    raises (access_denied, 
            not_found);
};
\end{verbatim}\end{minipage} & 
\begin{minipage}{7cm}\small\begin{verbatim}
/* DCE exception handling
   not supported */
   
   
   
   
   
   
\end{verbatim}\end{minipage} \\
\end{tabular}\end{center}

This means that the method \texttt{file::open} can, in addition to 
system exceptions, raise two user-defined exceptions named 
\texttt{access\_denied} and \texttt{not\_found}.

CORBA also allows exceptions to contain additional information; for example,
it may be useful to add a message for the user, or details on how to 
correct the error. Here is an example:

\begin{center}\begin{tabular}{l@{\hspace{.4cm}}|@{\hspace{.5cm}}l}
\begin{minipage}{7cm}\small\begin{verbatim}
interface file {
  exception access_denied {
    string reason;
    int missing_privileges;
  }

  void open(in string name)
    raises (access_denied);
};
\end{verbatim}\end{minipage} & 
\begin{minipage}{7cm}\small\begin{verbatim}
/* DCE exception handling 
   not supported */







\end{verbatim}\end{minipage} \\
\end{tabular}\end{center}

\emph{Note: Currently, \IDL only supports exceptions without additional
information}

\section{Constants}

It is possible to define constants within an interface. The constants
are added to the generated header files, but may also be used within
the specification itself. Consider the following example:

\begin{center}\begin{tabular}{l@{\hspace{.4cm}}|@{\hspace{.5cm}}l}
\begin{minipage}{7cm}\small\begin{verbatim}
interface foo {
  const int addr_size = 6;
  struct hwaddr {
    octet mac[addr_size];
  };
};
\end{verbatim}\end{minipage} & 
\begin{minipage}{7cm}\small\begin{verbatim}
interface foo {
  const int addr_size = 6;
  struct hwaddr {
    unsigned small mac[addr_size];
  };
};
\end{verbatim}\end{minipage} \\
\end{tabular}\end{center}

This causes a constant named \texttt{addr\_size} to be exported to the
client header file; also, the struct \texttt{hwaddr} is declared to be 
six bytes long.

\section{Attributes}

\subsection{Oneway functions}

Usually, a remote procedure call consists of two phases: A send phase,
in which the client sends a message to the server, and a receive phase,
in which it waits for a reply. However, in some cases, it is necessary
to omit the second phase, e.g. when the request is to be processed
asynchronously, or when no reply is possible. 

Here is how this behaviour can be specified:

\begin{center}\begin{tabular}{l@{\hspace{.4cm}}|@{\hspace{.5cm}}l}
\begin{minipage}{7cm}\small\begin{verbatim}
interface foo {
  oneway void bar(in int a);
};
\end{verbatim}\end{minipage} & 
\begin{minipage}{7cm}\small\begin{verbatim}
interface foo {
  [oneway] void bar([in] int a);
};
\end{verbatim}\end{minipage} \\
\end{tabular}\end{center}

Oneway methods must not have \texttt{inout} or \texttt{out} arguments;
also, they may neither return a value nor raise any exceptions.
Note that when the method returns on the client side, the absence of
exceptions does not mean that the request has been processed successfully;
it only indicates that the request transfer did not fail.

\subsection{Function identifiers}

As interfaces can contain multiple methods, and servers may implement
multiple interfaces, the server must be able to tell from the request
which method it is intended for. In \IDL, this is accomplished with
numeric function IDs. 

A function ID has two parts: An interface number and a method number.
The interface number is identical for all methods in an interface, whereas
different interfaces may be assigned the same number. The method number
must be unique within an interface.

By default, \IDL assigns the number 0 to all interfaces; this implicitly
assumes that different interfaces are implemented by different threads.
If this is not the case, you need to assign the interface numbers manually.
The allowed range for interface numbers is platform dependent; typically,
numbers of up to 1.000 are supported. Here is an example:

\begin{center}\begin{tabular}{l@{\hspace{.4cm}}|@{\hspace{.5cm}}l}
\begin{minipage}{7cm}\small\begin{verbatim}
[uuid(5)]
interface foo {
  void bar(in int a);
};
\end{verbatim}\end{minipage} & 
\begin{minipage}{7cm}\small\begin{verbatim}
[uuid(5)]
interface foo {
  void bar([in] int a);
};
\end{verbatim}\end{minipage} \\
\end{tabular}\end{center}

You can also change the assignment of method numbers by applying the
\texttt{uuid} attribute to individual methods. However, this is rarely
necessary.

\subsection{Mapping fpages}

The L4 IPC primitive supports mapping, which is essentially the transfer
of a complete memory page from one address space to another. As a result,
the page is shared by both address spaces; write operations in either of
them are instantly visible in both. Also, the page may cover the same
address range in both spaces, but this is not mandatory.

\IDL supports this primitive with a special data type, the \texttt{fpage},
which describes a memory region; see \cite{v2,x0,v4} for more details.
\texttt{fpage}s are implicitly mapped during message transfer; the mapping
persists and is not revoked upon completion of the call. Consider the
following example:

\begin{center}\begin{tabular}{l@{\hspace{.4cm}}|@{\hspace{.5cm}}l}
\begin{minipage}{7cm}\small\begin{verbatim}
interface pager {
  [uuid(0)]
  void pagefault(
    in int addr, in int ip,
    out fpage f
  );
};
\end{verbatim}\end{minipage} & 
\begin{minipage}{7cm}\small\begin{verbatim}
interface pager {
  [uuid(0)]
  void pagefault(
    [in] long addr, [in] long ip,
    [out] fpage *f
  );
};
\end{verbatim}\end{minipage} \\
\end{tabular}\end{center}

This defines a method \texttt{pagefault} which takes two arguments,
the fault address \texttt{addr} and the instruction pointer \texttt{ip};
the server replies with an fpage \texttt{f} which is to be mapped to 
the client address space.

\section{Advanced features}

\subsection{Local functions}

Some L4 microkernels support a special IPC primitive, the local IPC or
lipc, which is optimized for intra-address space calls \cite{lipc}. 
This feature can be used with \IDL as follows:

\begin{center}\begin{tabular}{l@{\hspace{.4cm}}|@{\hspace{.5cm}}l}
\begin{minipage}{7cm}\small\begin{verbatim}
local interface foo {
  short bar(
    in short a, 
    in short b
  );
};
\end{verbatim}\end{minipage} & 
\begin{minipage}{7cm}\small\begin{verbatim}
[local] interface foox {
  short bar(
    [in] short a, 
    [in] short b
  );
};
\end{verbatim}\end{minipage} \\
\end{tabular}\end{center}

The \texttt{local} attribute indicates that the methods in this interface
will \emph{only} be called via lipc. This permits \IDL to apply 
considerably more optimizations; for example, lipc will be used by both
client and server stubs, and parameters may be passed by reference.
Possible applications include semaphore servers or dispatcher threads,
which are used for distributing incoming requests to multiple worker threads
in the same address space.

\emph{Note: This is an experimental feature in the current release!}

\subsection{Type import from C++ code}

Usually, interface specifications should contain definitions for the data
types they use. However, it may sometimes be convenient to import additional
data types from the application code. \IDL supports this with a DCE-style
\texttt{import} statement:

\begin{center}\begin{tabular}{l@{\hspace{.4cm}}|@{\hspace{.5cm}}l}
\begin{minipage}{7cm}\small\begin{verbatim}
interface foo {
  import "l4/x86/types.h";
  
  void bar(in l4_taskid_t tid);
}; 
\end{verbatim}\end{minipage} & 
\begin{minipage}{7cm}\small\begin{verbatim}
interface foo {
  import "l4/x86/types.h";
  
  void bar([in] l4_taskid_t tid);
};
\end{verbatim}\end{minipage} \\
\end{tabular}\end{center}

This makes available the types defined in \texttt{l4/x86/types.h} to all
the methods in interface \texttt{foo}; \IDL contains a gcc-compatible
C++ parser for this purpose. At compile time, the header file is scanned, 
all global type definitions are imported and converted into IDL. 

Note that unlike types defined directly in IDL, the generated header files
do not contain definitions for imported types; instead, an \texttt{\#include}
directive referring to the original file is added.

\subsection{Disabling the memory allocator}

By default, \IDL uses CORBA-style dynamic memory allocation, i.e. it calls
\texttt{CORBA\_alloc} to reserve buffers for variable-length output values.
In some cases, however, the data is expected at a specific location, which
requires an additional copy on the client side. 

To avoid this problem, you can use the \texttt{prealloc} attribute.
Consider the following example:

\begin{center}\begin{tabular}{l@{\hspace{.4cm}}|@{\hspace{.5cm}}l}
\begin{minipage}{7cm}\small\begin{verbatim}
interface foo {
  typedef sequence<long, 100> buf;
  void bar(
    [prealloc] out buf x
  );
};
\end{verbatim}\end{minipage} & 
\begin{minipage}{7cm}\small\begin{verbatim}
interface foo {
  void bar(
    [out, prealloc, length_is(len)] 
    char **data, [out] short *len
  );
};
\end{verbatim}\end{minipage} \\
\end{tabular}\end{center}

No dynamic buffers are allocated in either case. Instead, you must
explicitly provide a buffer by initializing \texttt{x.\_buffer} or
\texttt{*data}, respectively. Also, you must supply the size of the
buffer in \texttt{x.\_maximum} (or \texttt{*len} in the DCE example).

While \texttt{prealloc} should be used for individual arguments,
a command line option is also available which disables dynamic allocation
for the entire file.

\subsection{Receiving kernel messages}

In the L4 world, exceptions are mapped to IPCs, which are sent by the kernel
on behalf of the faulting application, e.g. to its pager. These messages
can be received by \IDL stubs when special interface definitions are used.

The following interface handles X0-style page faults, which are e.g.
generated by the Hazelnut kernel:

\begin{center}\begin{tabular}{l@{\hspace{.4cm}}|@{\hspace{.5cm}}l}
\begin{minipage}{7cm}\small\begin{verbatim}
interface pager {
  [uuid(0)] void pagefault(
    in long addr,
    in long ip,
    out fpage p
  );
};
\end{verbatim}\end{minipage} & 
\begin{minipage}{7cm}\small\begin{verbatim}
interface pager {
  [uuid(0)] void pagefault(
    [in] long addr,
    [in] long ip,
    [out] fpage *p
  );
};
\end{verbatim}\end{minipage} \\
\end{tabular}\end{center}

\noindent Here is how a V4 page fault is specified (e.g. for the 
Pistachio kernel):

\begin{center}\begin{tabular}{l@{\hspace{.4cm}}|@{\hspace{.5cm}}l}
\begin{minipage}{7cm}\small\begin{verbatim}
interface pager {
  [kernelmsg(idl4::pagefault)] 
  void pagefault(
    in long addr,
    in long ip,
    in long privileges,
    out fpage p
  );
};
\end{verbatim}\end{minipage} & 
\begin{minipage}{7cm}\small\begin{verbatim}
interface pager {
  [kernelmsg(idl4::pagefault)] 
  void pagefault(
    [in] long addr,
    [in] long ip,
    [in] long privileges,
    [out] fpage *p
  );
};
\end{verbatim}\end{minipage} \\
\end{tabular}\end{center}

Note that in this case, the fault type (i.e. the requested privileges) is 
provided as a separate argument, whereas in the upper example, it is encoded
in the lower two bits of the fault address.
%\IDL also generates client stubs for those functions. These can e.g. be used
%to forward kernel messages to another thread.
